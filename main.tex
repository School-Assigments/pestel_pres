\documentclass[aspectratio=169]{beamer}

\usepackage[T2A]{fontenc}
\usepackage[utf8]{inputenc}
\usepackage[russian,english]{babel}
\usepackage{graphics}
\usepackage[normalem]{ulem}

\graphicspath{{res/}}
\DeclareGraphicsExtensions{.png,.jpg}

\usetheme{Copenhagen}

\title[Русская правда]{"Русская Правда" П.И.Пестеля}
\author[П.И.Пестель]{Чубий Савва\\Титов Максим\\Кулягина Дарья\\Касимова Есения}
\institute{Лицей НИУ ВШЭ}
\date{18.02.2021}

\begin{document}
\begin{frame}
    \maketitle
\end{frame}

\begin{frame}\frametitle{Введение}
    \begin{itemize}
        \item<1-> Общество
        \item<2-> Государство и его цель
        \item<3-> Повелевающие (правительство) и повинующиеся (народ)
        \item<4-> Первоначальная обязанность человека --- сохранение своего бытия
        \item<5-> Разделение власти
        \item<6-> Государственное благоденствия --- безопасность и благосостояние
        \item<7-> Необходимость Русской Правды
            и Временного Верховного Правительства
    \end{itemize}
\end{frame}

\begin{frame}\frametitle{Религия}
    \begin{itemize}
        \item<1-> Духовенство --- не сословие, но звание власти
        \item<2-> Духовенство --- белое и черное
        \item<3-> Иностранец --- не духовенство
        \item<4-> Духовные представители других религий вне России
        \item<5-> Духовные лицеи, медицина
        \item<6-> Содержание
    \end{itemize}
\end{frame}

\begin{frame}\frametitle{Территория. Границы}
    \begin{itemize}
        \item Север --- Ледовитое море
        \item Восток --- Большой океан
        \item Юг --- Китай, Саянские и Алтайские Горы, Туркеста, Туркеста/н, Бухария,
            Непроходимые пески между Каспийским и Аральским морями,
            Персия, Турция, Черное Море, Дунай и Валахия
        \item Запад --- Венгрия, Польша, Пруссия, Балтийское море,
            Ботнический Залив и Швеция
    \end{itemize}
\end{frame}

\begin{frame}\frametitle{Территория. Деление}
    \begin{itemize}
        \item<1-> 53 губернии
            \begin{itemize}
                \item<2-> 50 округа
                \item<3-> 3 удела
                    \begin{itemize}
                        \item Столичный
                        \item Донской
                        \item Аральский
                    \end{itemize}
            \end{itemize}
        \item<4-> 1 область = 5 округов
        \item<5-> 1 округ/удел = ? уездов
        \item<6-> 1 уезд = ? волостей
    \end{itemize}
\end{frame}

\begin{frame}\frametitle{Территория. Другое}
     \begin{itemize}
         \item<1-> Унитарное государство
         \item<2-> Частичная конфискация помещичьих земель
         \item<3-> Два земельных фонда:
             \begin{itemize}
                 \item Общественная земля
                 \item Частная земля
             \end{itemize}
        \item<4-> Нижний Новгород $\rightarrow$ Владимир --- столица\\
            Владимир $\rightarrow$ Клязмин
     \end{itemize}
\end{frame}

\begin{frame}\frametitle{Форма правления}
    \begin{itemize}
        \item<1-> Высшая законодательная власть --- Народное вече из 500 человек
        \item<2-> Верховная исполнительная власть --- Державная дума (5 человек на 5 лет)
        \item<3-> Высшая контрольная власть --- Верховный собор (120 человек, пожизненно)
        \item<4-> Распорядительная власть --- Наместные собрания
        \item<5-> Исполнительная власть --- Наместные правления
    \end{itemize}
\end{frame}

\begin{frame}\frametitle{Налоги}
    \begin{itemize}
        \item<1-> Прямые
            \begin{itemize}
                \item<2-> С имущества
                \item<3-> От чистой прибыли
                \item<4-> Распределение по областям в соотношении богатства
                \item<5-> В удобное для граждан время
            \end{itemize}
        \item <6-> Косвенные
            \begin{itemize}
                \item<7-> Минимум на необходимое
                \item<8-> Снижение цены на соль
            \end{itemize}
    \end{itemize}
\end{frame}

\begin{frame}\frametitle{Права и обязанности граждан}
    \begin{itemize}
        \item<1-> Жители
            \begin{itemize}
                \item Власть имеющие
                \item Свободные
                \item Зависимые
            \end{itemize}
        \item<2-> Посадские люди, смерды-общинники юридически и экономически независимы
        \item<3-> Посадское население:
            \begin{itemize}
                \item Боярство
                \item Духовенство
                \item Купечество
                \item "Низы"
            \end{itemize}
        \item<4-> Зависимые люди:
            \begin{itemize}
                \item Закупы
                \item Холопы
            \end{itemize}
    \end{itemize}
\end{frame}

\begin{frame}\frametitle{Мемы}
    \begin{columns}[onlytextwidth]
        \begin{column}{0.5\textwidth}
            \centering
            \includegraphics[height=0.8\textheight]{meme1}
        \end{column}
        \begin{column}{0.5\textwidth}
            \centering
            \includegraphics[height=0.8\textheight]{meme2}
        \end{column}
    \end{columns}
\end{frame}

\end{document}
